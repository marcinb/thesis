\chapter[Behavior Driven Development]{Behavior Driven Development}
  \section{Czym jest BDD?}
    Behavior Driven Development jest metodyką rozwoju oprogramowania, w której główny nacisk położony jest na zacieśnienie współpracy między programistami, a nie technicznymi uczestnikami projektu. Podobnie jak w przypadku TDD implementację konkretnej funkcjonalności poprzedza zdefiniowanie jej zachowania w teście. Różnica polega na tym, że scenariusze BDD pisane są w naturalnym języku, tak aby były zrozumiałe nie tylko dla programistów. Idealną sytuacją jest kiedy scenariusze takie powstają w wyniku ścisłej współpracy programistów oraz właścicieli projektu. BDD jak sama nazwa wskazuje kładzie nacisk na zdefiniowanie i zrozumienie zachowania aplikacji a szczegóły implementacji są tutaj mniej istotne.
    \subsection{BDD jako narzędzie dokumentacji. Funkcjonalności i scenariusze}
      Każdy program tak naprawdę składa się z zestawu funkcjonalności. Każda funkcjonalność wnosi konkretną wartość dodaną z punktu widzenia grupy docelowej, dla której oprogramowanie powstaje. Przykładowo program do obsługi księgowości powinien pozwalać zarządzać rachunkiem zysków i strat oraz sporządzać bilans (to oczywiście tylko niektóre z funkcji). Każda wyżej wymienionych czynności to pojedyncza funkcjonalność, która w zgodzie z filozofią BDD powinna zostać opisana zestawem scenariuszy.
      
      Scenariusze opisują zachowanie się programu w kontekście konkretnej funkcjonalności w precyzyjnie zdefiniowanej sytuacji a dla każdej funkcjonalności możemy zdefiniować dowolną ilość scenariuszy użycia. Na przykład dla funkcjonalności Zarządzanie rachunkiem zysków i strat mogą być zdefiniowane następujące scenariusze:
      
      \begin{itemize}
        \item Wprowadzenie nowej poprawnej pozycji
        \item Wprowadzenie nowej nie poprawnej pozycji
        \item Modyfikacja pozycji
        \item Próba wprowadzenia nowej pozycji przez nieautoryzowanego użytkownika
      \end{itemize}
      
      Narzędzia BDD pozwalają na bardzo dużą swobodę jeśli chodzi o język definiowania scenariuszy, przykładowy plik definiujący powyższą funkcjonalność wraz ze scenariuszem 'Wprowadzenie nowej nie poprawnej pozycji' mógłby wyglądać tak:
      
\lstinputlisting[language=ruby]{examples/code/ch03/01}

    Powyższy przykład opisuje zachowanie aplikacji w sposób na tyle naturalny, że nikt nie będzie miał trudności w jego zrozumieniu, a po poznaniu kilku prostych zasad językowych, którymi należy się kierować również nie techniczny uczestnicy projektu mogą opisywać nowe funkcjonalności w ten sposób.
    
    Tworzenie specyfikacji projektu w postaci scenariuszy BDD jest bardzo pożądane. Zmniejsza to ilość niepotrzebnej dokumentacji, powoduje zacieśnienie współpracy w zespole oraz, o czym więcej w następnym podrozdziale, zwiększa pokrycie kodu testami.
    
    \subsection{BDD jako narzędzie testowania}
      Prawdziwą mocą bibliotek wspierających Behavior Driven Development jest to, że traktują one specyfikacje dostarczoną w postaci scenariuszy jako zestaw testów. Dołączając więc scenariusze do naszego zestawu testów automatycznych mamy pewność, że oprogramowanie zachowuje się zgodnie z opisanymi w nich założeniami. W tym podrozdziale chciałbym opisać w jaki sposób, z technicznego punktu widzenia przebiega proces translacji specyfikacji (będącej plikiem tekstowym składającym się z pojedynczych scenariuszy) na zestaw automatycznych testów.
      \subsubsection{Definicje kroków na przykładzie biblioteki Cucumber}
        Aby skutecznie uruchomić scenariusze w formie testów należy skonstruować plik tekstowy je zawierający w zgodzie z pewnymi zasadami. Przykładowy plik definiujący funkcjonalność 'Zarządzanie rachunkiem zysków i strat' rządzi się pewnymi prawami:
        
        \begin{itemize}
          \item Nazwa opisywanej funkcjonalności zdefiniowana jest po słowie kluczowym 'Funkcjonalność:'
          \item Pojedyncze scenariusze definiowane są po słowie kluczowym 'Scenariusz:'. Należą do ostatnio zdefiniowanej funkcjonalności.
          \item Każda kolejna linia nie będąca definicją nowego scenariusza lub funkcjonalności traktowana jest jako pojedynczy krok ostatnio zdefiniowanego scenariusza.
        \end{itemize}
        
      Sercem biblioteki Cucumber są definicje kroków, które pozwalają powiązać każdy z nich z konkretną akcją. Definicja kroku składa się z wzorca językowego kroku oraz kodu który ma zostać wykonany jeśli wzorzec pasuje do aktualnie przetwarzanego kroku.
      
      Wzorzec językowy najczęściej przybiera postać wyrażenia regularnego. Cucumber przetwarzając nowy krok iteruje po wzorcach, które zostały zdefiniowane i wykonuje kod powiązany z pierwszym, do którego pasuje nazwa kroku (dlatego ważne jest definiowanie wzorców tak, aby były unikalne). Przykładowa definicja dla kroku 'Kiedy wybieram z menu głównego pozycję "Wprowadź nową pozycję"' może wyglądać tak:
      
      \lstinputlisting[language=ruby]{examples/code/ch03/02}
      
      Kiedy cucumber spróbuje przetworzyć krok 'Kiedy wybieram z menu głównego pozycję "Wprowadź nową pozycję"' dopasuje go do pierwszego odpowiadającego wzorca, który zdefiniowaliśmy oraz uruchomi blok kodu zdefiniowany pomiędzy słowami kluczowymi \verb+do+ oraz \verb+end+. Przekaże do tego bloku również wszelkie zmienne zdefiniowane we wzorcu. Powyższy przykład jest na razie bezużyteczny ponieważ jedyne co znajduje się w bloku kodu to komentarz.
      
      Krok scenariusza najczęściej definiuje jedną z trzech rzeczy: założenie co do stanu środowiska w jakim uruchomione jest oprogramowanie, konkretną akcję wykonywaną na oprogramowaniu (taką jak np. kliknięcie przycisku) lub rezultat, jakiego się spodziewamy. Cucumber sam w sobie jest narzędziem które dopasowuje krok do odpowiadającej mu definicji, wszelkie interakcje z działającym programem, modyfikowanie środowiska działania czy też testowanie otrzymywanych wyników muszą zostać wykonywane przez zewnętrzne biblioteki, które należy samodzielnie skonfigurować.
      
      Aby lepiej zobrazować w jaki sposób może wyglądać działające środowisko BDD poczynię kilka założeń co do aplikacji, którą testujemy, oraz bibliotek, których użyjemy. Pominę szczegóły konfiguracji, jest to treścią jednego z kolejnych rozdziałów tej pracy.
      
      \begin{itemize}
        \item Interfejs naszej aplikacji zdefiniowany jest językiem opisu dokumentów HTML.
        \item Aplikacja napisana jest w języku Ruby, przy pomocy frameworka Ruby on Rails
        \item Narzędzie RSpec zostało zainstalowane i skonfigurowane
        \item Narzędzie Capybara zostało zainstalowane i skonfigurowane
      \end{itemize}
      
      O bibliotece RSpec pisałem już w poprzednim rozdziale, tutaj użyjemy go w podobnym celu, to jest do testowania wyników działania aplikacji. Biblioteka Capybara służy do symulacji interakcji użytkownika z aplikacją używającą jako interfejsu HTML. Używając powyższych narzędzi mogę opisać pełen zestaw definicji kroków dla scenariusza 'Wprowadzenie nowej nie poprawnej pozycji':
      
      \lstinputlisting[language=ruby]{examples/code/ch03/03}
      
    \subsubsection{Rola programisty}
    
      Jak widać definiowanie kroków wymaga podstawowej wiedzy z zakresu programowania oraz budowy opisywanego systemu, dlatego też zadanie to najczęściej wykonują programiści pracujący nad projektem.  Klient w tym wypadku dostarcza specyfikacji w postaci scenariuszy użycia, programiści zaś definiują brakujące kroki. Na szczęście problem ten jest uciążliwy tylko na początku życia projektu. Wraz z rosnącą ilością scenariuszy rośnie też ilość definicji kroków, a w pewnym momencie życia projektu definicje kroków zaczynają gęsto pokrywać większość funkcjonalności systemu tak, że w większości wypadków nawet do nowych funkcjonalności można bez problemu dopasować istniejące już kroki.
      
      Ważne jest aby scenariusze były jak najbardziej zestandaryzowane, to znaczy, aby w miarę możliwości korzystać z już zdefiniowanych kroków w procesie ich tworzenia. Aby to osiągnąć osoba odpowiedzialna  za dostarczenie specyfikacji w postaci scenariuszy powinna poświęcić trochę czasu na zapoznanie się z dostępnymi definicjami kroków.
      
      Ze strony programistów, którzy piszą definicje szczególny nacisk powinien zostać położony na kilka kwestii:
      
      \begin{description}
        \item[Unikanie powtórzeń] Należy upewnić się, że brakujący krok jest na pewno unikalny. Może okazać się, że wcześniej został zdefiniowany bardzo podobny krok, o innej nazwie.
        \item[Parametryzacja] Jeśli zachodzi taka potrzeba należy przystosować definicję kroku tak, aby obsługiwała więcej niż jeden przypadek użycia.
        \item[Klasyfikacja] Aby zwiększyć czytelność kodu należy klasyfikować kroki na przykład według funkcjonalności, które obsługują, można w tym celu umieszczać kroki w osobnych plikach.
      \end{description}
      
      Przestrzeganie tych zasad nie jest konieczne do poprawnego działania systemu ale na pewno w znaczącym stopniu poprawi czytelność i jakość naszej bazy testów.
      
    \subsection{Testy akceptacyjne a BDD}
      
      Kiedy nowa funkcjonalność jest skończona klient zawsze powinien ją przetestować i zaakceptować jeśli spełnia jego oczekiwania, lub odrzucić w przeciwnym wypadku. W tradycyjnym podejściu do rozwoju oprogramowania testy akceptacyjne wyglądają najczęściej tak, że klient ręcznie sprawdza nową funkcjonalność. Minusy manualnego testowania zostały szczegółowo opisane w rozdziale dotyczącym TDD. Behavior Driven Development pozwala w dużym stopniu wyeliminować czasochłonne manualne testy akceptacyjne. W momencie, kiedy mamy gotowy dobry zestaw scenariuszy użycia, który jednocześnie jest specyfikacją funkcjonalności sam fakt, że wszystkie scenariusze zostają uruchomione z pozytywnym skutkiem jest wystarczającym testem akceptacyjnym.
      
      Jeśli chcemy zastąpić testy akceptacyjne z żywym klientem na scenariusze BDD należy pamiętać, że środowisko testowe (w tym definicje kroków) muszą zostać zaprojektowane tak, aby działały w identycznej konfiguracji jak konfiguracja produkcyjna (czyli taka, w jakiej działa nasze oprogramowanie po dostarczeniu do klienta). Szczególnie nie należy używać imitacji (ang. mock) obiektów, których bardzo często używa się w testach jednostkowych, scenariusze powinny testować zachowanie się aplikacji w naturalnym środowisku.
      
  \section{Narzędzia BDD dostępne dla języka Ruby}
    \subsection{Cucumber}
      W środowisku programistów języka Ruby najpopularniejszym narzędziem BDD jest bez wątpienia biblioteka Cucumber. Cucumber powstał w 2008 roku jako następca biblioteki Story Runner będącej częścią narzędzia RSpec. Biblioteka miała umożliwić programistom opisywanie naturalnym językiem sposobu w jaki powinno zachowywać się oprogramowanie. Scenariusze służą jednocześnie jako dokumentacja oraz zestaw automatycznych testów. 
      
      Niewątpliwą zaletą biblioteki Cucumber jest to, że możne ona współpracować również z innymi niż Ruby platformami programistycznymi takimi jak Java, .NET, Adobe Flex czy Python. Więcej informacji na ten temat znaleźć można pod adresem: http://github.com/aslakhellesoy/cucumber/wiki.
      
      Instalacja jest równie prosta jak w przypadku omawianej wcześniej biblioteki RSpec:
      
      \begin{verbatim}
        gem install cucumber
      \end{verbatim}
      
      Jako, że techniki prezentowane w tym rozdziale opisane i zaprezentowane zostały na przykładzie biblioteki Cucumber powtórny szczegółowy opis tego narzędzia jest w tym miejscu zbędny. Szczegółowa dokumentacja dostępna jest pod adresem: http://cukes.info
      
    \subsection{Steak}
      Steak podobnie jak Cucumber jest narzędziem, które służy do opisywania zachowania aplikacji w formie automatycznych testów, główny nacisk jest tutaj jednak kładziony na wygodę programisty oraz szybkość powstawania testów. Ograniczenia w stosunku do tego co oferuje Cucumber są następujące:
      
      \begin{itemize}
        \item Brak wsparcia dla naturalnego języka
        \item Brak podziału na scenariusze oraz kroki
        \item Konieczność pisania testów w Ruby
      \end{itemize}
      
      Steak rezygnuje z filozofii angażowania nie technicznych uczestników projektu w cykl testowania oprogramowania wychodząc z założenia, że opcja ta i tak nie jest wykorzystywana w większości projektów informatycznych. Programista sam konstruuje testy w języku Ruby, nie jest to tak czytelne jak w przypadku Cucumbera, jednak ich tworzenie i utrzymanie jest prostsze:
      
      \begin{itemize}
        \item Definicje kroków są zbędne
        \item Nie trzeba ustalać i trzymać się konwencji językowych
        \item Bezpośredni dostęp do wszystkich funkcji języka Ruby
      \end{itemize}
      
      Przykładowa specyfikacja napisana przy pomocy biblioteki Steak wygląda następująco:
      
      \lstinputlisting[language=ruby]{examples/code/ch03/04}
      
      Instalacja jak zwykle w przypadku (ogromnej większości) bibliotek Ruby sprowadza się do
      
      \begin{verbatim}
        gem install steak
      \end{verbatim}
      
      W przypadku, kiedy nasza aplikacja oparta jest o framework Rails, Steak oferuje dodatkowe udogodnienia, ale wymaga również minimalnie więcej konfiguracji. Wszelkie kroki są jasno opisane pod adresem http://github.com/cavalle/steak
      
      Steak jest dobrym rozwiązaniem jeśli projekt prowadzony jest przez stricte techniczny zespół lub zarząd projektu nie jest przekonany do uczestniczenia w procesie specyfikowania oprogramowania w formie testów. Minusem rezygnacji z takiej formy współpracy jest niestety spadek znaczenia tworzonej w Steaku specyfikacji jako testów akceptacyjnych: jeśli właściciele projektu nie będą rozumieli specyfikacji w formie jaką oferuje to najprawdopodobniej testy akceptacyjne będą chcieli przeprowadzić osobiście.
    
    \subsection{Capybara}  

      Capybara ułatwia integracyjne testowanie aplikacji działających w przeglądarce www symulując akcję jakie wykonywałby żywy użytkownik. Nie jest to narzędzie związane stricte z BDD, jednak świetnie uzupełnia i wspomaga biblioteki takie jak Cucumber czy Steak.
      
      Chcąc opisać krok po kroku zachowanie aplikacji internetowej, naturalnym jest łączenie akcji, które wykonuje w przeglądarce użytkownik z ich konsekwencjami. Przykładowo w celu zalogowania się należy najpierw wypełnić pole tekstowe 'Login', następnie wpisać hasło do pola 'Hasło' oraz ostatecznie kliknąć przycisk 'Zaloguj'. Capybara pozwala symulować interakcje użytkownika z naszą aplikacją w przeglądarce i wykonywać na niej takie akcje jak kliknięcie w link, wypełnienie pola czy odświeżenie strony.
      Informacją zwrotną, która często służy do określenia efektu naszych działań jest obiekt \verb+page+, który przechowuje ostatnią wyświetloną użytkownikowi stronę. Przykładowo kod:
      
      \lstinputlisting[language=ruby]{examples/code/ch03/05}
      
      Wypełni wszelkie konieczne do zalogowania się pola, zasymuluje kliknięcie przycisku z etykietą 'Sign in' oraz sprawdzi, czy w wyniku tego, na nowo wyświetlonej stronie znajduje się treść 'Successfuly logged in!'.
      
      Instalacja podstawowej biblioteki odbywa się poprzez:
      
      \begin{verbatim}
        gem install capybara
      \end{verbatim}
      
      Tak jak w przypadku Cucumbera, funkcje Capybary wykorzystywane były w tym rozdziale w celu zobrazowania założeń BDD. Dokładna dokumentacja bardzo bogatych możliwości biblioteki wykracza poza ramy tej pracy, można ją jednak odnaleźć wraz ze szczegółami konfiguracji dla różnych środowisk na stronie http://github.com/jnicklas/capybara.
      