\chapter[Studium przypadku: Dynamicznie generowany panel administracyjny]{Studium przypadku: Dynamicznie generowany panel administracyjny}
  \section{Wstęp}
  Świat technologii informatycznych zmienia się bardzo szybko. Ogromnemu skokowi mocy obliczeniowej sprzętu komputerowego towarzyszył w ostatnich kilku latach znaczący spadek cen związanych z jego wykorzystaniem. Stało się to katalizatorem rozwoju nowych trendów między innymi w dziedzinie wytwarzania oprogramowania. Gwałtowny wzrost zainteresowania dynamicznymi językami programowania takimi jak Ruby, Python czy JavaScript oraz pojawienie się nowych metodyk rozwoju produktów informatycznych, które coraz częściej ujmują ten proces bardziej z filozoficznego aniżeli technicznego lub biznesowego punktu widzenia jest bez wątpienia jednym z owoców tego procesu.
  
  Spadek kosztów mocy obliczeniowej skutecznie rozwiązał problem wyboru technologii realizacji projektu informatycznego: wydajność narzędzi, których użyjemy do realizacji celu stała się w większości przypadków pomijalnym lub przynajmniej drugorzędnym problemem. Dziś najważniejszym kryterium wyboru jest stopień dopasowania możliwości oraz charakterystyki rozważanej technologii do potrzeb zespołu odpowiedzialnego za rozwój projektu. Oczywiście istnieją również skrajne przypadki, w których to oprogramowanie, z różnych przyczyn, musi zostać napisane w konkretnej technologii. Te smutne przypadki stanowią jednak kroplę w morzu wykraczającą daleko poza ramy niniejszej pracy.
  
  W czasach, kiedy o wiele bardziej opłaca się dokupić nowy serwer niż pokrywać koszty optymalizacji ogromną furorę robi termin \"Przedwczesna optymalizacja\". Wydaje się to dziwne, ale te dwa słowa mają dzisiaj znaczenie negatywne, które jest jednak jak najbardziej uzasadnione z ekonomicznego oraz użytkowego punktu widzenia. Dopóki niedostatki w wydajności oprogramowania można skompensować inwestycją w nowe zasoby sprzętowe zespół programistów powinien skupiać wszystkie swoje wysiłki na rozwój funkcjonalności. Optymalizacja kodu następuje dopiero w momencie, kiedy koszty inwestycji w sprzęt przewyższają koszt związane z optymalizacją albo w momencie kiedy oprogramowanie ze względu na swoją nieoptymalnośc przestaje się skalować na nowe zasoby sprzętowe.
  
  Na pierwszy rzut oka może wydawać się, że ten wstęp niewiele ma wspólnego z tematem pracy. Należy jednak uświadomić sobie, że to właśnie opisane powyżej zmiany w sposobie myślenia o metodach prowadzenia projektów IT stoją u podstaw rozwoju nowoczesnych narzędzi takich jak wymienione wcześniej dynamiczne języki, wysokopoziomowe frameworki programistyczne na nich oparte czy metodologie pokroju Behaviour Driven Development. Środki, które służą osiągnięciu założonego celu są jedną z najważniejszych zmiennych od których zależy sukces projektu, w przeszłości istniało wiele barier ograniczających ich wybór, dziś większość z nich została usunięta.
  
  \section{Założenia projektu}
  
  
  \section{Proces implementacji}
  \section{Wnioski}
