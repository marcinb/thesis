\chapter[Studium przypadku: Dynamicznie generowany panel administracyjny]{Studium przypadku: Dynamicznie generowany panel administracyjny}
  \section{Wstęp}
  Świat technologii informatycznych zmienia się bardzo szybko. Ogromnemu skokowi mocy obliczeniowej sprzętu komputerowego towarzyszył w ostatnich kilku latach znaczący spadek cen związanych z jego wykorzystaniem. Stało się to katalizatorem rozwoju nowych trendów między innymi w dziedzinie wytwarzania oprogramowania. Gwałtowny wzrost zainteresowania dynamicznymi językami programowania takimi jak Ruby, Python czy JavaScript oraz pojawienie się nowych metodyk rozwoju produktów informatycznych, które coraz częściej ujmują ten proces bardziej z filozoficznego aniżeli technicznego lub biznesowego punktu widzenia jest bez wątpienia jednym z owoców tego procesu.
  
  Spadek kosztów mocy obliczeniowej skutecznie rozwiązał problem wyboru technologii realizacji projektu informatycznego: wydajność narzędzi, których użyjemy do realizacji celu stała się w większości przypadków pomijalnym lub przynajmniej drugorzędnym problemem. Dziś najważniejszym kryterium wyboru jest stopień dopasowania możliwości oraz charakterystyki rozważanej technologii do potrzeb zespołu odpowiedzialnego za rozwój projektu. Oczywiście istnieją również skrajne przypadki, w których to oprogramowanie, z różnych przyczyn, musi zostać napisane w konkretnej technologii. Te smutne przypadki stanowią jednak kroplę w morzu wykraczającą daleko poza ramy niniejszej pracy.
  
  W czasach, kiedy o wiele bardziej opłaca się dokupić nowy serwer niż pokrywać koszty optymalizacji ogromną furorę robi termin \"Przedwczesna optymalizacja\". Te dwa słowa mają dzisiaj znaczenie negatywne, które jest jednak jak najbardziej uzasadnione z ekonomicznego oraz użytkowego punktu widzenia. Dopóki niedostatki w wydajności oprogramowania można skompensować inwestycją w nowe zasoby sprzętowe zespół programistów powinien skupiać wszystkie swoje wysiłki na rozwój funkcjonalności. Optymalizacja kodu następuje dopiero w momencie, kiedy koszty inwestycji w sprzęt przewyższają koszt związane z optymalizacją albo w momencie kiedy oprogramowanie ze względu na swoją nie optymalność przestaje się skalować na nowe zasoby sprzętowe.
  
  Na pierwszy rzut oka może wydawać się, że ten wstęp niewiele ma wspólnego z tematem pracy. Należy jednak uświadomić sobie, że to właśnie opisane powyżej zmiany w sposobie myślenia o metodach prowadzenia projektów IT stoją u podstaw rozwoju nowoczesnych narzędzi takich jak wymienione wcześniej dynamiczne języki, wysokopoziomowe frameworki programistyczne na nich oparte czy metodologie pokroju Behaviour Driven Development. Środki, które służą osiągnięciu założonego celu są jedną z najważniejszych zmiennych od których zależy sukces projektu, w przeszłości istniało wiele barier ograniczających ich wybór, dziś większość z nich została usunięta.
  
  \section{Założenia projektu}
  Celem niniejszego rozdziału jest pokazanie czytelnikowi w jaki sposób rozwija się konkretny projekt prowadzony w zgodzie z metodologią BDD oraz jakie płyną z tego korzyści. Wybrany temat projektu to dynamiczny panel administracyjny dla aplikacji internetowych opartych na bibliotece Ruby on Rails, jego główne założenia to:
  
  \begin{description}
    \item[Uniwersalność] Panel powinien współpracować z większością aplikacji napisanych w Ruby on Rails. W praktyce oznacza to, że jeśli modele biznesowe powinny być klasami pochodnymi klasy \verb+ActiveRecord+ a sposób budowy aplikacji jest zgodny z konwencjami przyjętymi dla aplikacji Ruby on Rails.
    \item[CRUD] W teorii baz danych istnieją cztery podstawowe operacje jakie możemy wykonać na zasobie: Tworzenie, odczytanie, aktualizacja, usunięcie (ang. Create, read, update and delete). Biblioteka ma pozwalać na zarządzanie modelami biznesowymi aplikacji przy użyciu jedynie tych czterech standardowych metod.
    \item[Dynamiczność] Po instalacji panel powinien sam wykryć rodzaje zasobów na jakich operuje aplikacja oraz wygenerować odpowiednie widoki i formularze do zarządzania nimi. Zmiany w budowie modeli biznesowych, które wymuszają konieczność zmian w zarządzaniu nimi, jak również pojawienie się nowych modeli również powinno być odzwierciedlone w zachowaniu panelu automatycznie, bez konieczności jakiejkolwiek ingerencji.
    \item[Proste wdrożenie] Podstawowe wdrożenie rozwiązania wymaga jedynie aby aplikacja kliencka korzystała z biblioteki Ruby on Rails w wersji co najmniej 3.0.3. Po dodaniu biblioteki panelu do listy gemów wykorzystywanych przez aplikację możliwe jest natychmiastowe korzystanie.
    \item[Wygoda użytkowania] Proces zarządzania aplikacją powinien być jak najwygodniejszy. Oznacza to między innymi, że pola formularzy służących do tworzenia lub edycji rekordów powinny być dostosowane do rodzaju danych jakie przechowują a próby wprowadzenia nieprawidłowych wartości powinny być sygnalizowane czytelną informacją o błędzie. Jeśli istnieją powiązania pomiędzy kilkoma modelami biznesowymi, to powinna istnieć bardzo szybka możliwość zarządzania każdym z powiązanych rekordów.
  \end{description}
  
  \subsection{Open Source}
    Biblioteka jest dostępna za darmo na zasadach licencji MIT.\footnote{http://en.wikipedia.org/wiki/MIT\_License} Filozofia rozwoju oprogramowania na zasadach open source jest bardzo bliska środowisku programistów Ruby. Sam język udostępniony jest na licencji GPL, biblioteka Ruby on Rails korzysta z licencji MIT. Użycie licencji MIT oznacza, że każdy otrzymuje prawo do nielimitowanego wykorzystania kopii oprogramowania w dowolny sposób, sprawia to, że jest to najchętniej wykorzystywana przez programistów Ruby licencja.
    
  \subsection{Sposób prowadzenia projektu}
  Projekt prowadzony jest według bardzo uproszczonych zasad metodologii SCRUM.\footnote{http://en.wikipedia.org/wiki/Scrum\_(development)} Rozwój projektu podzielony jest na tygodniowe sprinty, przed każdym z nich następuje spotkanie zespołu, podczas którego wybierane i przydzielane są konkretne zadania do wykonania w następnej iteracji. Spotkania te służą również omówieniu bieżących spraw związanych z projektem.
  
  Zespół zaangażowany w projekt składa się z trzech osób: dwóch programistów, oraz osoby dzielącej rolę Scrum Mastera, który odpowiedzialny jest za przygotowanie i prowadzenie spotkań oraz Product Ownera, który reprezentuje oczekiwania końcowego użytkownika dotyczące kwestii funkcjonalności oraz ekonomicznych kwestii związanych z rozwojem projektu.
  
  Duży nacisk kładziony jest na testowanie oprogramowania, testy akceptacyjne istnieją w formie zautomatyzowanych scenariuszy BDD. Każda funkcjonalność lub modyfikacja oprogramowania akceptowana jest jedynie jeśli dostarczona jest wraz z pełnym zestawem testów ją dokumentujących.
  
  \subsection{Dodatkowe narzędzia}
  Repozytorium projektu zarządzane jest przez system kontroli wersji GIT\footnote{http://git\-scm.com} a hostowane jest przez serwis GitHub.\footnote{http://github.com} Źródła projektu dostępne są publicznie pod adresem https://github.com/piotrj/administer.
  
  Jako narzędzie wspomagające proces zarządzania projektem użyta została darmowa wersja Pivotal Tracker\footnote{http://www.pivotaltracker.com}, który został zaprojektowany aby wspomagać zarządzanie projektem prowadzonym według zasad SCRUM.
  
  \section{Proces implementacji}
  \section{Wnioski}
