% Copyright 2011, Piotr Jakubowski

\chapter{Implementation}
  The aim of this chapter is to provide description of how the plugin codebase,
  show how it has been designed and what decisions has been made. In addition it aims to indicate
  interesting examples of usage of Ruby and Ruby on Rails API.
  
  \section{Overview}
    The plugin has been developed as Rails Engine (details will be covered in next chapter). This
    means that it is a Rails application that can be embedded into another one. The fact, that 
    the plugin is a Rails application provides the ability of using whole power of Ruby on Rails.
    So, it is possible to use the MVC pattern, set up routing for the application and so on.
    
    In order to get the plugin running following parts had to be developed:
    \begin{itemize}
      \item Setting the application as \texttt{Rails Engine} as said before.
      \item Handling the \texttt{Models} - finding all models in the application as well as
        getting the model class from string representation
      \item Getting the list of \texttt{Fields} in the model along with their type and any
        necessary attributes that would be necessary for generating the form for given model.
        This includes not only fields such as text or integer fields, but also handling associations
        between models which is the most challenging task.
      \item Elaborate a convenient way of defining \texttt{Configuration} for the plugin that would be both flexible
        and easy to use at the same time.
      \item Develop the web part of the plugin - setting up the \texttt{Routing, Controllers and Html Templates} that 
        would allow users to smoothly use the application.
    \end{itemize}
    
    The plugin has been developed along and tested with a simple application that would act like a simple blogging
    system. Appropriate features such as particular types of model associations has been added to the test application
    when needed in order to see whether the plugin handles it in correct way. As test application is not integral 
    part of the plugin it would not be described in details. In short, the models that would appear in the test application are
    following:
      \begin{description}
        \item[Posts] used as main model to test all associations on. It has standard fields such as string field for title,
          text field(long string) for body, date field for published at date.
        \item[Categories] can be attached to Post in association one-to-many. Used to test belongs\_to association in Post.
        \item[Comments] attached to Post in many-to-one association. Used to test has\_many association.
        \item[Attachment] attached to Post in one-to-one association. Used to test has\_one association.
      \end{description}
      
    Association testing has been shown from the Post model perspective, but the plugin would handle the associations
    from both ends as it needs to be universal and be working for every application.
    
  \section{Rails Engine}
    ToDo
  \section{Models}
    ToDo
  \section{Fields}
    ToDo
  \section{Configuration}
    ToDo
  \section{Routing, Controllers and Html Templates}
      