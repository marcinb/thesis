% Copyright 2011, Piotr Jakubowski

\chapter{Research}
  The aim of this chapter is to present the environment in which the plugin will be developed. I will present, firstly, the features of Ruby language and then what is Ruby on Rails, how it is constructed and how we can enrich it with plugins. In the end I will take a look into possibilities of metaprogramming in Ruby and Rails.
  
  \section{Ruby}
  Here you will find information abut Ruby language and especially those features that create the power of Ruby and distinguish it from other programming languages.
    \subsection{Basic information}
      \subsubsection{History}
      Ruby language has been created by Yukihiro Matsumoto also known as "Matz" for english speaking programmers. It firstly appeared publicly in 1995 with version 0.95 in order to reach version 1.0 a year later. As of the time of writing this document, the latest release of Ruby is of branch 1.9 (specifically 1.9.2), but the branch 2.0 with brand new exciting features is emerging on the horizon.
    
      While creating Ruby Matz stated that he focused on developing a language that would be programmer-friendly and that would make the barrier between idea and putting this idea in code as low as possible. The most famous quote of Matz fully describes the philosophy of the language:
    
      \begin{quote}
        Ruby is designed to make programmers happy
      \end{quote}
    
      That explains the idea behind the project.
    
      \subsubsection{Implementation}
      The official implementation of Ruby has been written in C language. As no particular standard has been developed for the language, the official implementation is a reference for all other vendors. Nonetheless, there is plenty of other implementations including one operating in JVM(JRuby\footnote{http://www.jruby.org/}), .NET framework(IronRuby\footnote{http://ironruby.codeplex.com/}) and Objective-C runtime(MacRuby\footnote{http://www.macruby.org/}).
    
    \subsection{Features}
      \subsubsection{Basic features}
      It is worth mentioning few basic features of Ruby language that would make the further part of this chapter clearer.
      \begin{itemize}
        \item Ruby is scripting language - statements are executed as provided and there is no special \texttt{main} function
    	  \item Ruby is dynamically typed
    	  \item Method invocations do not need to include parentheses
    	\end{itemize}
    	
      \subsubsection{Everything is an object}
      Ruby is an object oriented language. In fact, every value in Ruby is a object. Even such "primitive" values as integers, true, false or nil (which is Ruby's NULL). Thanks to that, Ruby promotes and enforce object oriented programming. 
      
      Moreover, you can have methods on integers or other primitive values which makes the code much more readable, shorter and enjoyable. For example instead of having Java-like:
      
      \lstinputlisting[language=java]{code/chapter03/objects.java}
      
      You can have much more readable:
      
      \lstinputlisting[language=ruby]{code/chapter03/objects.rb}
      
      Ruby standard library does not include \texttt{to\_roman} function for integers, but we could easily add one by using next described feature.
      
      \subsubsection{All classes are open}
      Ruby gives us freedom to change already declared classes. Moreover, it does not mean that we can change only the classes that we defined. We can change even classes defined by Ruby standard library. Therefore we can open for example class Integer that represents all integer values in Ruby code and add method \texttt{to\_roman}:
      
      \lstinputlisting[language=ruby]{code/chapter03/openclasses.rb}
      
      Now, we would be able to have following line in our code:
      
      \lstinputlisting[language=ruby]{code/chapter03/openclasses2.rb}
      
      \subsubsection{Power of blocks}
      This is one of the most exciting features of Ruby language. Block is a fragment of code that can be passed to a function and the function may call this code anywhere inside its body. 
      
      Block in ruby can be denoted in two ways:
      
      \lstinputlisting[language=ruby]{code/chapter03/blocks01.rb}
      
      Usually the first way is used for blocks that span throughout multiple lines, while the second way is used for single line blocks. 
      
      Blocks can be used in number of ways. The most popular are iterators:
      
      \lstinputlisting[language=ruby]{code/chapter03/blocks02.rb}
      
      Above code would result in printing 1, 2, 3 to the output. Statement \texttt{do |i|} means that this block expects one argument called \texttt{i} just like regular functions do.
      
      In order to understand how blocks operate, let me present simple example:
      
      \lstinputlisting[language=ruby]{code/chapter03/blocks03.rb}
      
      The above example adds the \texttt{each\_nested} method to an array that allows us to perform operations on every element of multidimensional matrix. It iterates over every element of an array. If given element is again an array then it recursively calls \texttt{each\_nested} on it and if it is other element then it passes the control to a block along with the element as a parameter.
      
      As we can see appropriate use of blocks may be very useful and can make code much shorter and more readable
      
    \subsection{Summary}
    This chapter demonstrated basic features of Ruby language. In order to get more information on this language I forward the reader to the official website\footnote{http://ruby-lang.org/} or to some books from bibliography\cite{ruby01}.
    
  \section{RubyGems}
    \subsection{Basic Information}
  RubyGems\footnote{http://rubygems.org/} is the packaging system to distribute libraries for Ruby language. The packages and libraries are called gems and console front-end script is called (not surprisingly) \texttt{gem} (distributed along with ruby 1.9, for ruby 1.8 needs to be installed separately). With the help of RubyGems Ruby program can be easily enhanced our with variety of functionalities (for instance serializing/deserializing JSON, handling weekends and holidays in Dates, authentication system etc). 
  
  In order to use particular gem first it has to be installed with \texttt{gem} command 
  \begin{lstlisting}
    gem install some_gem
  \end{lstlisting}
  
  And then it has to be added to application by first requiring rubygems itself and then requiring the gem.
  
  \begin{lstlisting}
    require 'rubygems'
    require 'some_gem'
  \end{lstlisting}
  
  With those few lines of code it is possible to add really extensive features to Ruby programs, as the community gets bigger and bigger and, moreover, its members are willing to share the code.
  
    \subsection{Dependency management}
  Of course, every gem can use other gems in order to build up on their functionalities. However, in order for those gem to function properly all their dependencies. Moreover, different version of the same gem can differ not only in the implementation, but also in their API. Therefore gems need to keep track of their versions. Surely, it would be totally inefficient if users would have to take care of this on their own. This is why RubyGems take care of it for us.
  
  Every gem is described by its gemspec (gem specification) file. Gemspec is simply Ruby code that defines particular gem. Below is presented gemspec for Ruby on Rails:
  
  \lstinputlisting[language=ruby]{code/chapter03/rails.gemspec}
  
  As seen above, gemspec is pretty straightforward. Along with other attributes, developer can specify dependencies of his gem. Then, when user installs it RubyGems traverses the list of dependencies and installs all of them (of course in the meantime traversing their gemspecs and installing theirs dependencies, which results in the entire tree of dependency). That way, RubyGems becomes really efficient tool to handle libraries. 
  
  \section{Ruby on Rails}
    \subsection{Introduction}
  Ruby on Rails is a comprehensive framework for web applications development. It has been created by David Heinemeier Hansson in 2004 and a year later he earned Google and O'Reilly's Hacker of the Year award. Its main characteristic is promoting convention over configuration, what enables developers to quickly build even complex web apps without the need to write lengthy XML configuration files. It utilizes Model-View-Controller pattern, which happens to be very good choice for web applications and helps produce clean and maintainable code.
  
  Ruby on Rails is distributed as a gem for Ruby language. After just one command (provided that Ruby and RubyGems are installed on the system):
  
  \begin{lstlisting}
    gem install rails
  \end{lstlisting}
  
  the user has entire environment for creating web applications (along with development server).
  
  Among others, Ruby on Rails has two main parts:
  \begin{description}
    \item[ActiveRecord] Object-relational mapper
    \item[ActionPack] Framework for handling requests and rendering responses
  \end{description}
  
  This two parts mainly create the power of the framework and will be described in more details later.
  
    \subsubsection{Structure of the application}
    As mentioned before, Ruby on Rails takes its power and agility from "convention over configuration" rule. Therefore, Ruby on Rails applications have very well defined directory structure. The user can easily generate one for his application by using following command:
    \begin{lstlisting}
      rails new name_of_application
    \end{lstlisting}
    
    Or in case the user is using older (below 3.0) version of Rails:
    
    \begin{lstlisting}
      rails name_of_application
    \end{lstlisting}
    
    This command generates the entire directory tree that is used in Rails application. Most important directories are described below.

    \begin{description}
      \item[app] Contains main code of applications. It is divided into models, controllers and views subfolders
      \item[config] Holds configuration files
      \item[db] Holds files that regard database
      \item[lib] Contains files that create environment for the application but are considered as strictly connected to business logic of application
      \item[public] Place for assets of the application: static pages, stylesheets, javascript files - anything that can be statically served.
    \end{description}

    Thanks to such well-defined structure, Ruby on Rails knows exactly where to look for particular parts of the system. Moreover, it does not need to go through all the files, but rather load them on demand.

    \subsection{ActiveRecord}
      \subsubsection{Introduction}
      ActiveRecord is one of Ruby's Relational Object Mappers, but has been developed as
      part of Ruby on Rails framework. As whole framework, ActiveRecord philosophy is convention
      over configuration.

      Usually, models in Rails applications are stored in app/models directory. They are classes that
      derive from ActiveRecord::Base class. And, in fact, this is enough to handle simple models.
      It is possible to create persistent business models with great ease.

      \lstinputlisting[language=ruby]{code/chapter03/activerecord01.rb}

      Above example depicts the most simple model. Of course, the user needs to set up configuration
      for database connection. Usually it is done in config/database.yml file while in Ruby on Rails project
      or provide paramaters directly to ActiveRecord (if not used in Rails project).

      \lstinputlisting[language=ruby]{code/chapter03/activerecord02.rb}

      \subsubsection{Basic Model}
      Coming back to the code example of model:

      \lstinputlisting[language=ruby]{code/chapter03/activerecord01.rb}

      This will map automatically to items table in the database. It is not necessary to specify
      what kind of columns, the table has. All attributes of the model are automatically
      drawn from the table and (thanks to dynamic nature of ruby) translated into appropriate
      accessor methods for Ruby class.

      So assuming the table has been created with following sql statement

       \lstinputlisting[language=ruby]{code/chapter03/activerecord01.rb}

      Then with the ruby code presented above we get following functionallity:

       \lstinputlisting[language=ruby]{code/chapter03/activerecord04.rb}

      Moreover, it is possible to search for records by forming (even very complex) queries:

      \lstinputlisting[language=ruby]{code/chapter03/activerecord05.rb}

      \subsubsection{Associations}
      As every Object Relational Mapper, ActiveRecord enables developers to specify relations
      between models. The only thing that needs to be done in order to handle associations well
      is add the foreign key column to the database and describe the relation in the model by
      using one of three class methods:
        \begin{itemize}
          \item{has\_many}
          \item{has\_one}
          \item{belongs\_to}
        \end{itemize}
      This is shown in the example:

      \lstinputlisting[language=ruby]{code/chapter03/activerecord06.rb}

      \subsubsection{Validations}
      As models usually keep great deal of application's business logic there is another feature
      that is very useful - validations. It is possible to define rules against which the
      model will be tested. If some of those conditions would not be fulfilled, the record would
      not be saved and ActiveRecord would store Errors object that gathers information on
      unsatisfied validations.

      \lstinputlisting[language=ruby]{code/chapter03/activerecord07.rb}

      Functionality presented above is just a small subset of possibilities. It is possible
      to specify when given validations should be executed (during create or update of the record)
      or specify conditions that turn off validation (eg. do not check presence of one attribute
      if the second one is true). Moreover, except for standard \texttt{validates\_*} functions,
      the user can specify his own validations by passing the names of methods that would be
      executed during validations to \texttt{validate} class method.

      \subsubsection{Migrations}
      Among others, ActiveRecord provides another feature that makes the life of developers
      just a little less painful - Migrations. They are used to handle changes to the database,
      and as an extension - to structure of Models. 

      So, instead of creating and altering tables from SQL we could use a migration that looks
      like this:

      \lstinputlisting[language=ruby]{code/chapter03/activerecord08.rb}

      Rails provide tools that make work with migrations like generators that help create migration
      files or tools for running migrations (\texttt{rake db:migrate}). Moreover, in the applicaiton
      database Rails would create a table called \texttt{schema\_migrations} that keeps track of 
      which migrations have been applied to given database. This is useful, when working on some
      project with fellow developers - if one of them adds a migration and then the other developer
      pulls it from the version control system, he can run \texttt{rake db:migrate} and Rails
      would run only new migrations.

      \subsubsection{Summary}
      As stated in the introduction, it is clear to observe, that ActiveRecord provides great deal
      of features that surely makes the rapid development of applications a breeze. Moreover, thanks
      to Ruby's nature it all is packed in very easy and accessible API.

      The intent of this chapter was to make a short introduction to ActiveRecord and it did not cover
      all of interesting and useful features of ActiveRecord. There is plenty of other things such as
      callbacks, observers, named\_scopes that are worth noting, but in order to keep
      this document as focused on the topic as possible they are omitted here. In order to get more
      information, the reader can refer to ActiveRecord documentation.

