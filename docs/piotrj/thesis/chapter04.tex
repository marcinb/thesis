% Copyright 2011, Piotr Jakubowski
\nocite{rspec}
\nocite{testing}
\nocite{agile}

\chapter{Process and methodologies of solution development}
  \section{Creating a RubyGem}
    The first step in creating the plugin is setting up the structure of directories and creating necessary
    files for this plugin to be available for installation. Chapter 3.2 provided basic information about
    the RubyGems standard. This chapter will provide some further details on how the structure for particular
    RubyGem has been created.

    \subsection{Structure of directories}
      The essential structure of directories needed in RubyGem standard is a \texttt{gemspec} file in the
      root directory of the Gem and the \texttt{lib/} directory containing Ruby file with the name
      corresponding to the name of the Gem. So in the case of the Gem described in the application,
      which would be called Administer, the name of the file would be \texttt{lib/administer.rb}. This
      file is automatically loaded when the Gem is added to the program. You could put all the code
      in that file, but as the project gets bigger it is common practice to split the code into multiple
      files and require those files in the main (\texttt{administer.rb}) file.

      Instead of creating the structure manually, there are plenty of options that would help create
      the structure automatically. The one to be considered the best recently is using bundler, which is
      a gem for managing other gems in the application. Bundler among others, provides \texttt{bundle gem}
      command which creates structure needed for the newly created Gem. Moreover, it initializes \texttt{git}
      repository in the Gem directory for Version Control:

      \lstinputlisting{code/chapter04/gem01}

    \subsection{Gemspec}
      The \texttt{bundle gem} command creates a template for gemspec file:

      \lstinputlisting{code/chapter04/gem02}

      The gemspec file defines the specification of the gem. In addition to the attributes visible
      above (which are self-explanatory) there are few other that may be pretty important:
      \begin{description}
        \item[add\_dependency] Adds other gem as dependency
        \item[add\_development\_dependency] Adds ohter gem as dependency for development
      \end{description}

    \subsection{Gem building}
      When the gem's code is ready it is necessary to create a package. It can be achieved by running

       \begin{lstlisting}
          gem build gem_name.gemspec
        \end{lstlisting}

      This results in creating \texttt{.gem} package, which is a binary file.

    \subsection{Gem publishing}
      It is possible to distribute gem by sending the \texttt{.gem} file to other people via email or
      putting in on ftp server. But, there is unified way of distributing gems which is
      \texttt{rubygems.org} server.

      Publishing gem on \texttt{rubygems.org} is as easy as running

      \begin{lstlisting}
        gem push name_of_package.gem
       \end{lstlisting}

      where \texttt{name\_of\_package.gem} is a file we got after building the gem. In order to be able to
      push to rubygems developer has to have an account and will be prompted for login credentials.
  \section{Scrum}
    In order to ensure that the works both on Master Thesis and on the plugin go smoothly and according to
    plan I decided to use SCRUM methodology to lead the project. As SCRUM is usually intended for
    multi-person teams, it needed to be slightly adapted for the needs of the project of this Master Thesis.
    Nonetheless, it helped to evenly distribute workload in time and avoid exhausting bursts of intensive work
    when some deadline was approaching.

    \subsection{What is Scrum?}
      The idea behind Scrum is to divide the entire project into very small subprojects that can be done in
      intervals of up to 4 weeks. Such intervals are called \texttt{sprints}. The result of sprint is
      supposed to be part of the project that can be added to the production system. So most likely, it would
      be a complete feature that can be developed independently of other pending features and added to the
      existing system, what would accomplish the \texttt{sprint}.

      Before the project kicks off, the target functionality is discussed and decided. The functionality is
      then divided into features that would together build the product. Next, the difficulty of every feature
      is discussed and estimates on needed time and resources are drawn. This step is very important, as
      all next steps in the project would depend on this initial one.

      Each sprint would begin with a short team meeting with a purpose of drawing appropriate number
      of features from the backlog that would be implemented during that particular sprint. As it can be
      observed, the estimates from the initial step would become useful at this step as the team
      knows how much work would approximately it take to finish particular feature. Then the features are
      prioritized.

      During the sprint, developers take the features from the top of sprint's backlog and try to finish
      all of them before the end of the sprint. As the features has been selected to fit the workforce of the
      team for particular sprint, this should be doable without overtimes and stressful situations.

      Moreover, some implementations of Scrum introduce very short daily meetings, a purpose of which is
      to keep everyone in the team informed about the status of the work done as well as discussing problems
      that appeared.

      The sprint ends with a team meeting that reviews the performance, what has been done and what could not
      be completed. In addition, problems are discussed and conclusions are drawn for the future sprints.

    \subsection{Scrum for Master Thesis}
      As the Master Thesis should be and usually is a single person effort not all parts of Scrum can be
      implemented for the purpose of creating one. Nonetheless, the idea of dividing the whole project
      into very small and short steps helped me to create the plan of work. In order to
      embrace the power of meetings as motivational tool I contacted one of Ruby on Rails companies located
      in Łódź in order to organise weekly meeting with Ruby on Rails mentor. He would weekly review my
      work, help me with solving programming problems and guide into the world of Ruby programming best practices.
      In addition, about every month I tried to contact my supervisor in order to share and discuss my progress.

      As for planning work for my weekly sprints, I used Pivotal Tracker\footnote{http://pivotaltracker.com}
      to keep track of all tasks that needed to be done in order to get closer to finishing the project.
      In the begining, when starting the Master Thesis, I decided how the programming project should look like and what
      would be the parts of the report I would hand in and I would sort all of those issues according to
      what should be done first. That would create my backlog. Then every week I dragged appropriate number of tasks
      to the "current" section and at the end of the week I would get asked about.

      Not surprisingly, such arrangement helped to keep me motivated throughout the entire process and
      helped me avoid stress of deadlines and the feeling of not knowing where to go next.

  \section{Testing}
    \subsection{Introduction}
      The community gathered around Ruby language and especially framework Ruby on Rails is strongly
      focused on creating software using test oriented methodologies. There are two results of such positive peer pressure.
      First, most of applications and libraries that appear in the Ruby ecosystem come up with set of comprehensive
      test suite that can instantly confirm correctness of the application. Second, there are a lot of tools
      that developers can choose from in order to best satisfy their taste and needs.
    
    \subsection{Advantages of automatic tests}
      Software can be tested either manually or automatically. Manual testing seems to be a good
      choice at first as it seems you can test how the application interacts with real human and 
      how it reacts to his input. However, after longer consideration, few drawbacks of such method
      can be pointed out:
        \begin{itemize}
          \item If developers are used as testers they waste their time clicking through the application while
          they could could have been doing productive work on the code.
          \item If someone else is testing the project he needs firstly to spend a lot of time on 
          getting to know the project assumptions and aims.
          \item It is really hard to manually test every possible input and situation for every new feature.
          \item It does not give possibility of testing the details of implementations.
          \item Developers cannot get instant feedback
        \end{itemize}
      Moreover, those disadvantages rise along with the growth of the project. 
      
      That is why more and more projects introduces software that automatically tests the application and
      that way diminishes the need of manual testing. One of advantages of this is the fact that not only
      it is possible to test the end-to-end operation of the application but also test the internal implementation
      by using \texttt{unit tests} for testing individual classes or methods and \texttt{integration tests} which
      test the integration between components of the application.
      
      Automatic testing provides several advantages:
        \begin{itemize}
          \item Ability to run the test suite on every build of application
          \item Instant notice when some part of the application (even remote from developer is currently working on)
          fails
          \item Ability to test components of application in isolation
          \item Possibility of setting up appropriate state of application environment in order to test
          every possible situation
        \end{itemize}
      
    \subsection{Test Driven Development}
    Test Driven Development (also abbreviated as TDD) is one of test oriented methodologies of software development. Its philosophy is
    summarized as \texttt{Red-Green-Refactor-Repeat}, which means:
      \begin{description}
        \item[Red] Write a test for new feature that should not pass yet (usually denoted by red color)
        \item[Green] Write the most simple code that would make the test pass (denoted by green color)
        \item[Refactor] Rewrite the code so it is well-designed and easy to maintain while still keeping the test passing
        \item[Repeat] Repeat those steps for next feature
      \end{description}
      
      One of most important rules of TDD is that the test should be written before any code. That way, the developer
      has to first understand the feature he needs to add and then write the code. Such approach results in higher
      productivity.
      
      Test Driven Development suite usually consists of two types of tests: unit tests and integration tests.
      
      \subsubsection{Unit Tests}
        Unit tests focus on testin one of components of code in isolation. This means that all components
        that communicate with the component which is a subject of test, would get substituted by pseudo-implementations that give
        predictable results. So for example for code like this:

          \lstinputlisting[language=ruby]{code/chapter04/testing01.rb}

        If the developer would be testing \texttt{method\_a} from class \texttt{A} we would substitute implementation of 
        \texttt{method\_b} from class \texttt{B} with implementation that gives a result the developer expects. Such action
        is called mocking or stubbing. That way, the behavior of \texttt{method\_a} is tested under known circumstances as it is not 
        important at the moment how \texttt{method\_b}. In one of Ruby testing frameworks RSpec the test for \texttt{method\_a} could
        look like that:
      
          \lstinputlisting[language=ruby]{code/chapter04/testing02.rb}
      
        Thanks to such approach if test fails it is instantly clear which part of the system behaves in a wrong way without
        the need for lengthy debugging.
      
      \subsubsection{Integration Tests}
        The reason of having integration test is to see whether the components in fact cooperate together as supposed.
        As unit test focus on testing in isolation, some messages that come from other parts of the system may be not
        taken into account by the developer so it is important to test the whole system to see whether everything is fine.
        
    \subsection{Behavior Driven Development}
      Another test oriented methodology is Behavior Driven Development (abbreviated as BDD). The idea behind this is to 
      first define the behavior of the application for given feature and then write the code needed for that feature to be working.
      It sounds a lot like TDD, but the main difference is that the test is written for the interaction with end user - it does not
      take into account details of implementation, but rather how the application behaves and interacts via user interface. Moreover,
      those tests are usually written in language which is close to natural. This is caused by the fact, that Behavior Driven
      Development aims to drag non-technical part of the team into process of testing the application. By defining tests in language
      that may be understood by anyone we enable managers and product owners to write test cases or reading them in order to 
      verify that the test case in fact satisfies the business need.
      
      Below example of test written for Cucumber (which is one of Ruby's BDD tool) shows how such tests can be written:
      \lstinputlisting{code/chapter04/testing03.feature}
      
    
    \subsection{Tools available for Ruby on Rails}
      As said before the test-oriented community produced great amount of libraries, frameworks and tools for testing. The most
      important ones are:
        \begin{description}
          \item[Test::Unit] Default test framework for Ruby 1.8
          \item[MiniTest] Successor of Test::Unit, default for Ruby 1.9
          \item[RSpec] Test framework that aims to substitute Test::Unit or MiniTest. It provides more BDD-like syntax
            and has an incorporated mocking framework.
          \item[Cucumber] BDD tool that enables developers to specify their features using
            Gherkin\footnote{https://github.com/cucumber/cucumber/wiki/Gherkin} language.
          \item[FactoryGirl] Tool that aims to substitute fixtures which are used by default in test frameworks.
          \item[Capybara, Webrat] Tools for simulating user interaction with application via web browser
          \item[Steak] Integrates Rspec and Capybara and makes it possible to write end-to-end tests in pure Rspec instead 
            of Cucumber
        \end{description}
      
    \subsection{Summary}
      After considering the ideas and tools I have decided to use a mixture of TDD and BDD in order to grow my application.
      All new features would be first described in terms of BDD in order to understand exactly how the application should
      act. Then, while writing code unit test would be written in order to understand what is expected from the implementation
      and in order to make the design modular and testable which leads to code that is easier to read and maintain.
      
      The tools that will be used are following: Rspec, Capybara, FactoryGirl and Capybara.
    
      
    
        
    