\chapter{Conclusions}
  As mentioned in chapter \ref{ch:implementation:general_outcome} about the outcome of the project 
  (page \pageref{ch:implementation:general_outcome}) the project turned out to be successful.
  
  One of my biggest concerns while I was starting the project was
  that the application needs to be universal and 
  usable for all kind of purposes. Fortunately, it turned out that if enough of
  work and study is put into this, it is not only doable but also very enjoyable.
  
  The plugin is far from being fully complete. Although it has all the functionalities
  that were set up in the beginning, it still can get a lot of improvements. 
  Probably, the state when the developer says that this tool is total and does not
  need anything else in fact never happens. Needs of people would change and therefore
  new feature requests would emerge. Moreover, Rails framework is still evolving
  and the only fact of keeping the plugin up to date with the newest 
  versions of the framework would take a lot of work.
  
  Fortunately, the code base has been developed in a way
  that should be easily extendable. The fact that the development of the
  plugin has been driven by tests forced the design of code so that it 
  is easy to change in one place without breaking it somewhere else. 
  And if something breaks it would draw immediate attention as some
  of the test would fail.
  
  There are already few features
  that probably would get added to the plugin in the nearest feature 
  (as described in chapter \ref{ch:implementation:unsolved_problems} p. 
  \pageref{ch:implementation:unsolved_problems}). I intend to make 
  this plugin as comprehensive as possible. 
  
  As I would start to spreading the word about the plugin throughout the community,
  I hope for a lot constructive criticism and feature requests. Hopefully, this project
  would start to fulfill needs of Ruby on Rails developers and would make their life 
  easier.
    
  